\documentclass[a4paper,11pt]{article}
\usepackage[spanish]{babel}
\selectlanguage{spanish}
\spanishdecimal{.} % Si se quiere definir el punto decimal como punto en vez de coma
\addto\captionsspanish{\def\tablename{Tabla}} % Para renombrar todos los 'Cuadro' con 'Tabla'.
\usepackage{fancyhdr}
\usepackage[dvips]{graphicx}
\usepackage[normal]{caption2}
\usepackage{amsfonts,amssymb,amsmath,amsthm}
\usepackage{moreverb}
\usepackage{url}
\usepackage[T1]{fontenc}
\pagestyle{fancy}

\pagestyle{fancy}
\lhead{Criptograf�a y Seguridad}
\chead{}
\rhead{ITBA}
\cfoot{\thepage}
\renewcommand{\footrulewidth}{0.4pt}

\newcommand{\expon}[1]{~10^{#1}}
\newcommand{\dif}[1]{\mathrm{d}#1}

% \\			Line break
% \today		Fecha del sistema
% \label{}		Etiqueta
% \ref{}		Referencia a la etiqueta del mismo nombre (a la secci�n)
% \footnote{}		Nota al pie de la p�gina
% \underline{}		Subrayado
% \emph{}		Cursiva
% \textbf{}		Negrita
% \mbox{}		No separa lo que est� entre corchetes
% \begin{}		Comienza el entorno entre corchetes
% \end{}		Termina el entrorno entre corchetes
% \thepage		N�mero de p�gina
% \url{}		URL
% \section{}		Secci�n
% \subsection{}		Subsecci�n
% \subsubsection{}	Subsubsecci�n
% \frac{a}{b}		Fracci�n (a sobre b)
% a_t			A sub t (expr. matem�tica)
% $a$			Letra matem�tica a (puede ser por ejemplo \tau o directamente una letra

% Para enumerar:
% \begin{itemize}
% \item Enumerado 1.
% \item Enumerado 2.
% \end{itemize}

% Para poner eps:
% \begin{figure}
% \begin{center}
% \includegraphics[scale=0.6]{Nombre del eps}
% \caption{Lo que va a decir abajo}
% \label{Para hacerle referencia}
% \end{center}
% \end{figure}

% Para hacer ecuaci�n:
% \begin{equation}
% \label{Para hacerle referencia}
% Expresion
% \end{equation}

% Para hacer ecuaci�n alineando un igual (por columnas)
%\begin{eqnarray*}
%\label{Para hacerle referencia}
%X &=& 3 + 7\\
%&=& 10
%\end{eqnarray*}

% Para hacer tabla:
% \begin{table}
% \begin{center}
% \begin{tabular}{|c|c|} \hline Formato: cada pipe es una l�nea vertical.
% a & b \\ \hline Formato: cada \hline es una l�nea horizontal.
% \end{tabular}
% \caption{Nombre de la tabla.}
% \label{Para hacerle referencia}
% \end{center}
% \end{table}


\title{{\bf Modelado de Amenazas del sistema de voto electr�nico \emph{Monkeyvote}}}
\author{M. Besio \and P. Garc�a \and E. G�mez Balaguer \and H. Rajchert}
\date{\today}

\begin{document}

\maketitle
\thispagestyle{fancy}

\section{Introducci�n}
El objetivo de este trabajo es descubrir las potenciales amenazas y las vulnerabilidades presentes en la aplicaci�n de voto electr�nico \emph{Monkeyvote}\footnote{Se puede acceder a su c�digo fuente de forma an�nima mediante SVN en \url{http://monkeyvote.googlecode.com/svn/trunk}.}. El an�lisis es realizado considerando las mismas condiciones y supuestos para las que fue desarrollado el sistema.

\section{Modelado del sistema y del entorno}

\subsection{Entry Points}

\subsection{Assets}

\subsection{Trust Levels}
Naturalmente, los niveles de confianza ser�n 4, y corresponder�n a cada uno de los entes en que se divide el sistema. Dado que los servidores tienen un funcionamiento completamente aut�nomo se les puede asignar a cada uno encargados de su administraci�n y asociarles un nivel de confianza. Por otro lado, los votantes tambi�n conforman un trust level separado. En la tabla \ref{trust_levels} se apreciar los niveles de confianza con un identificador y una descripci�n asociada.

 Las actividades para cada nivel de confianza ser�n muy distintas y por ninguna raz�n deber�an mezclarse. Este es un sistema donde los l�mites est�n muy bien marcados, y como la seguridad es un aspecto m�s que importante aqu�, todo aquello que los trasgreda es un posible causal de malfuncionamiento.

\begin{table}
\begin{center}
\begin{tabular}{|c|c|c|} \hline
\textbf{ID} & \textbf{Nombre} & \textbf{Descripci�n}\\ \hline
1 & Votante & Persona registrada en el sistema\\
&& para votar.\\ \hline
2 & Administrador de \verb#Auth1# & Encargado de la administraci�n del\\
&& servidor de autenticaci�n 1.\\ \hline
3 & Administrador de \verb#Auth2# & Encargado de la administraci�n del\\
&& servidor de autenticaci�n 2.\\ \hline
4 & Administrador de \verb#Votos# & Encargado de la administraci�n del\\
&& servidor de votos.\\ \hline
\end{tabular}
\caption{Trust levels: Divisi�n en niveles de confianza para los participantes del sistema.}
\label{trust_levels}
\end{center}
\end{table}

\subsection{Escenarios de uso}

\subsection{Dependencias externas}

\subsection{Notas de seguridad}

\subsection{DFD del sistema}



\begin{thebibliography}{99}
\bibitem{MSDN}\url{http://msdn.microsoft.com/library/default.asp?url=/library/en-us/dnpag2/html/tmwa.asp}.
\end{thebibliography}

\newpage

\tableofcontents

\begin{center}
\tiny{\emph{Made with \LaTeX}}
\end{center}

\end{document}

